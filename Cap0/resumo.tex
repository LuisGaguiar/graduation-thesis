

%Futebol de robôs humanoides é uma atividade bem competitiva que visa ampliar os limites da pesquisa em robótica. Um dos diversos desafios envolvidos em jogar futebol é o desafio de chutar a bola eficientemente, de acordo com cada situação ao longo da partida. Além disso, temos presenciado um avanço continuo nas técnicas recentes em Deep Reinforcement Learning para aprender complexos problemas de controle em espaços de estados contínuo, como problemas de locomoção e de diversos movimentos presentes em robótica. DRL livre de modelo é um campo da grande área de aprendizado de máquina que combina algoritmos de aprendizado por reforço (RL) com métodos de aprendizado supervisionado (SL) utilizando deep neural-networks. DRL se adequa muito bem a problemas de locomoção em robótica, uma vez que elimina a necessidade de modelar a complexa dinâmica de um robô humanoide. Nesse trabalho, nós focamos no específico problema de ensinar um robô humanoide a chutar uma bola em direção a uma distância final planejada. Primeiramente, esse documento apresenta uma descrição do problema, acompanhada dos trabalhos relacionados e da abordagem proposta pelos autores, em seguida, é fornecida uma introdução às teorias relacionadas a aprendizado por reforço e a aprendizado supervisionado, finalizando com uma descrição sucinta das técnicas mais recentes em DRL. No futuro, planejamos alcançar um comportamento completo na ação mencionada, através do uso de algoritmos em DRL para inicialmente aprender um comportamento básico a partir da imitação de um movimento de chute existente, e então desenvolver o comportamento de forma aprender por reforço a chutar a bola até uma distância final desejada.

%Humanoid robot soccer is a very traditional competitive task that aims to push boundaries of state-of-the-art in robotics. One of the many challenges of playing soccer is kicking a ball efficiently, according to each context during the game. Besides, we have been witnessing the improvement in modern Deep Reinforcement Learning (DRL) techniques towards the goal of learning complex continuous control problems such as locomotion and several others movements present in robotics. Model-free DRL is a Machine-Learning field that combines Deep Learning (DL) methods with Reinforcement Learning (RL) algorithms, and greatly fit robotics locomotion problems since it avoids dealing with complex dynamics of a humanoid robot. In this work, we focused on a particular problem of the humanoid robot soccer domain which consists in teaching the robot to kick a ball towards a final planned distance. This document first gives a presentation of the given problem, with the related works and proposed approach, followed by a introduction of some background concerning RL and DL, ending with a summarized description of some cutting-edge DRL techniques. In the future, we plan to achieve a full behavior for the intended action, by using DRL techniques in first learning a initial behavior, using imitation learning on an already built kick movement, and then learning how to achieve the ultimate goal of reaching a desired final distance for the ball.